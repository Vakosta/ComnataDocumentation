\intro{Терминология}{1}

\noindent
\textbf{HLS (HTTP Live Streaming)} -- протокол для потоковой передачи медиа, разработанный компанией Apple.\\\\
\textbf{HTTP} -- протокол передачи данных, работающий в режиме <<запрос-ответ>>.\\\\
\textbf{IoC (Inversion of Control)} -- архитектурный принцип объектно-ориентированного программирования, используемый для уменьшения зацепления (связанности) в компьютерных программах.\\\\
\textbf{WebSocket} -- протокол поверх TCP-соединения, позволяющий взаимодействовать клиенту и серверу в режиме реального времени.\\\\
\textbf{База данных} -- совокупность данных, хранимых в соответствии со схемой данных, манипулирование которыми выполняют в соответствии с правилами средств моделирования данных.\\\\
\textbf{Клиент} -- приложение, которое подключается к серверу.\\\\
\textbf{Комната} -- виртуальное пространство, в котором воспроизводится видео и к которому могут подключаться пользователи.\\\\
\textbf{Реакция} -- всплывающий смайлик для быстрой передачи эмоций во время просмотра.\\\\
\textbf{Терминал} -- разновидность текстового интерфейса между человеком и компьютером, в котором инструкции компьютеру даются в основном путём ввода с клавиатуры текстовых строк (команд).\\\\
\textbf{Сегмент} -- небольшой отрывок из исходного видео.\\\\
\textbf{Сервер} -- выделенный или специализированный компьютер для выполнения сервисного программного обеспечения.\\\\
\textbf{Серверное приложение} -- программа, запущенная на сервере.\\\\
\textbf{СУБД} -- система управления базами данных.\\\\
\textbf{Фреймворк} -- компонент к программе, который включает в себя дополнительный реализованный функционал, упрощающий разработку.\\\\