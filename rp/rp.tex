\documentclass{../includes/TechDoc}
\usepackage[T1]{fontenc}
\usepackage[utf8]{inputenc}
%\usepackage[pdftex]{graphicx}
\DeclareGraphicsRule{*}{mps}{*}{}
\documentclass[12pt]{article}
\usepackage{tabularx}
\usepackage{listings}
\usepackage[russian,british]{babel}
\usepackage[utf8]{inputenc}
\setcounter{tocdepth}{4}

\renewcommand{\cftsecleader}{\cftdotfill{\cftdotsep}}

\newcommand{\intro}[1]{
    \stepcounter{section}
    \section*{\hfillПРИЛОЖЕНИЕ \arabic{section}}
    \begin{center}
        \Large\bf{#1}
    \end{center}
    \markboth{\MakeUppercase{#1}}{}
    \addcontentsline{toc}{section}{Приложение \arabic{section}. #1}
}

\lstset{basicstyle=\ttfamily,
    showstringspaces=false,
    commentstyle=\color{red},
    keywordstyle=\color{blue}
}

\title{Приложение для совместного просмотра фильмов}
\author{Студент группы БПИ-194}{В. А. Анненков}
\academicTeacher{Старший преподаватель департамента программной инженерии факультета компьютерных наук}{А. В. Поповкин}

\documentTitle{Руководство программиста}
\documentCode{RU.17701729.02.06-01 81 01-1}

\begin{document}
    \maketitle

    \begin{abstract}
        В данном программном документе приведено руководство программиста по использованию сервера <<Приложения для совместного просмотра фильмов>>.

        В разделе <<Назначение программы>> указаны назначение и функции, выполняемые программой.

        В разделе <<Условия выполнения программы>> указаны условия, необходимые для выполнения программы (объём памяти на жёстком диске, требования к составу и параметрам периферийных устройств, требования к программному обеспечению и т.п.).

        В разделе <<Обращение к программе>> приведено описание основных характеристик и особенностей программы.

        Настоящий документ разработан в соответствии с требованиями:
        \begin{enumerate}
            \item ГОСТ 19.101-77 Виды программ и программных документов;
            \item ГОСТ 19.102-77 Стадии разработки;
            \item ГОСТ 19.103-77 Обозначения программ и программных документов;
            \item ГОСТ 19.104-78 Основные надписи;
            \item ГОСТ 19.105-78 Общие требования к программным документам;
            \item ГОСТ 19.106-78 Требования к программным документам, выполненным печатным способом.
            Требования к содержанию и оформлению.
        \end{enumerate}
    \end{abstract}

    \newpage

    \tableofcontents

    \section{Назначение программы}

    \subsection{Назначение программы}

    Программа представляет из себя централизованное API, которое могут использовать другие клиенты с целью реализации функционала для совместного просмотра фильмов. Программа состоит из четырёх сервисов, два -- функциональные, другие два предназначены для объединения функциональных микросервисов в единую систему.

    Первый сервис отвечает за общее взаимодействие с клиентской частью с помощью протокола WebSocket. С помощью этого сервиса клиентская часть может осущетсвлять обмен сообщениями в чате, реакциями, а также синхронизировать видео между различными клиентами. Кроме этого, данный сервис осуществляет всё взаимодействие с серверной базой данных.

    Второй сервис отвечает за обработку и хранение видео. На этот сервис загружаются все видео, после чего конвертируются в формат HLS.

    \subsection{Информация о функциях и принципе эксплуатации программы}

    Программа обеспечивает возможность выполнения перечисленных ниже функций:
    \begin{enumerate}
        \item Загрузка видеофайла для его дальнейшей обработки в формат HLS.
        \item Корректное получение обработанного видео в формате HLS.
        \item Получение актуальных данных о перемотке видео с других клиентов.
        Поддержание видео в актуальном состоянии.
        \item Получение актуальных данных о сообщениях в чате.
        \item Получение актуальных данных о <<реакциях>>.
    \end{enumerate}

    \section{Условия выполнения программы}

    \subsection{Требования к составу и параметрам технических средств}

    \begin{itemize}
    	\item[--] Доступ к сети интернет.
    	\item[--] Как минимум 1 ГБ свободного места для корректного скачивания видео с сервера.
	\end{itemize}

    \subsection{Требования к информационной и программной совместимости}

    Программа должна поддерживать протоколы связи HTTP и WebSocket.

    \section{Обращение к программе}

    \subsection{Описание процедур вызова программы}

    \subsubsection{Описание общего взаимодействия через протокол WebSocket}

    Каждому клиенту требуется установить соединение с WebSocket сервером по адресу <</main/ws>>.
    После этого клиенты могут подписаться на специальные <<топики>> комнат.
    Таким образом, когда в <<топике>> будет появляться новое действие (например, новое сообщение или новая <<реакция>>), это действие получит любой клиент, подписанный на этот <<топик>>.

    \paragraph{Прослушивание команд от сервера}
    Для прослушивания действий доступны следующие <<топики>>:
    \begin{itemize}
    	\item[--] \textbf{<</topic/room/\{room\_id\}/videoActions>>} -- Действия о воспроизведении/паузе/перемотке видео.

    	Параметры объекта:
    	\begin{itemize}[noitemsep]
    		\item[--] actionId -- уникальный идентификатор действия;
    		\item[--] author -- клиент, совершивший действие;
    		\item[--] seekTime -- время в видео на этапе совершения действия;
    		\item[--] type -- тип действия (воспроизведение/пауза/перемотка);
    		\item[--] step -- если требуется дополнительное подтверждение -- <<CHECK>>, при полной готовности -- <<READY>>;
    		\item[--] actionTime -- общие дата и время, когда было совершено действие.
    	\end{itemize}
    	\item[--] \textbf{<</topic/room/\{room\_id\}/chatMessages>>} -- Новые сообщения в чате.

    	Параметры объекта:
    	\begin{itemize}[noitemsep]
    		\item[--] userId -- клиент, который отправил сообщение в чат;
    		\item[--] text -- текст сообщения;
    		\item[--] dateTime -- общие дата и время, когда было отправлено сообщение.
    	\end{itemize}
    	\item[--] \textbf{<</topic/room/\{room\_id\}/reactions>>} -- Новые <<реакции>>.

    	Параметры объекта:
    	\begin{itemize}[noitemsep]
    		\item[--] userId -- клиент, который отправил реакцию;
    		\item[--] type -- тип реакции;
    		\item[--] dateTime -- общие дата и время, когда была отправлена реакция.
    	\end{itemize}
    	\item[--] \textbf{<</topic/room/\{room\_id\}/joins>>} -- Присоединение новых пользователей к комнате.

    	Параметры объекта:
    	\begin{itemize}[noitemsep]
    		\item[--] newUserId -- клиент, который подключился к комнате;
    		\item[--] allUserIds -- массив всех клиентов, подключённых к комнате в данный момент, включая нового;
    		\item[--] dateTime -- общие дата и время, когда пользователь подключился к комнате.
    	\end{itemize}
    	\item[--] \textbf{<</topic/room/\{room\_id\}/lefts>>} -- Выход пользователей из комнаты.

    	Параметры объекта:
    	\begin{itemize}[noitemsep]
    		\item[--] leftUserId -- клиент, который вышел из комнаты;
    		\item[--] remainingUserIds -- массив всех клиентов, подключённых к комнате в данный момент, исключая вышедшего;
    		\item[--] dateTime -- общие дата и время, когда пользователь покинул комнату.
    	\end{itemize}
    \end{itemize}

    \paragraph{Отправка команд серверу}
    Серверу можно также отправлять запросы. Тогда их получат все клиенты, которые подписаны на <<топики>> соответствующей комнаты. Для отправки доступны следующие действия:
    \begin{itemize}
    	\item[--] \textbf{<</app/room/\{room\_id\}/videoAction>>} -- Действия о воспроизведении/паузе/перемотке видео.

    	Параметры объекта:
    	\begin{itemize}[noitemsep]
    		\item[--] seekTime -- время в видео на этапе совершения действия;
    		\item[--] type -- тип действия (воспроизведение/пауза/перемотка).
    	\end{itemize}
    	\item[--] \textbf{<</app/room/\{room\_id\}/videoActionReady>>} -- Подтверждение действия о воспроизведении/паузе/перемотке видео. Требуется отправлять, когда плеер загрузил фрагмент видео и способен его воспроизвести.

    	Параметры объекта:
    	\begin{itemize}[noitemsep]
    		\item[--] actionId -- уникальный идентификатор действия, к которому требуется подтвердить готовность.
    	\end{itemize}
    	\item[--] \textbf{<</app/room/\{room\_id\}/chatMessage>>} -- Отправка сообщения в чат.

    	Параметры объекта:
    	\begin{itemize}[noitemsep]
    		\item[--] text -- текст сообщения.
    	\end{itemize}
    	\item[--] \textbf{<</app/room/\{room\_id\}/reaction>>} -- Отправка <<реакции>>.

    	Параметры объекта:
    	\begin{itemize}[noitemsep]
    		\item[--] reaction -- тип реакции.
    	\end{itemize}
    	\item[--] \textbf{<</app/room/\{room\_id\}/join>>} -- Уведомить о подключении к комнате.
    \end{itemize}

    \subsubsection{Описание загрузки видеофайла}

    Для загрузки видеофайла на сервер требуется отправить файл по адресу <</video/upload>>.

    При корректном запросе будет получен ответ со следующими полями:
    \begin{itemize}
    	\item[--] status -- <<SUCCESS>> или <<ERROR>>.
    	\item[--] videoId -- уникальной идентификатор видеоролика и комнаты.
    	\item[--] videoUrl -- ссылка для воспроизведения видеоролика.
    \end{itemize}


    \subsection{Входные и выходные данные}

    \subsubsection{Описание и обоснование метода организации входных данных}

    Взаимодействие с клиентами организовано в виде HTTP и WebSocket подключений.

    При подключении через HTTP формат входных данных зависит от типа HTTP-метода:
    \begin{enumerate}
        \item GET -- входные данные задаются в url.
        \item POST, PUT, DELETE -- параметры задаются в теле запроса.
    \end{enumerate}

    При передаче входных данных через WebSocket формат входных данных представляет из себя JSON-объект.

    \subsubsection{Описание и обоснование метода организации выходных данных}

    При взаимодействии с любым протоколом формат выходных данных -- JSON объект.

    \subsubsection{Примеры запросов}

    \paragraph{Протокол HTTP}

    \noindentОтправлен корректный запрос с видеофайлом на адрес \url{https://comnata.tv/video/upload};\\

    \noindentПолучен ответ:
    \begin{lstlisting}[language=json,caption={Пример ответа при работе с HTTP протоколом}]
    {
  		"status": "SUCCESS",
  		"videoId": "322e60eedf4245a08ebc1d9bfc23a5bd",
  		"videoUrl": "/video/getVideo/322e60eedf4245a08ebc1d9bfc23a5bd/video.m3u8"
	}
    \end{lstlisting}

    \paragraph{Протокол WebSocket}

    \begin{lstlisting}[language=json,caption={Пример запроса при работе с WebSocket протоколом}]
    >>> SEND
	destination:/app/room/B73EAD/chatMessage
	content-length:14

	{"text":"Hello World!"}
    \end{lstlisting}

    \begin{lstlisting}[language=json,caption={Пример ответа при работе с WebSocket протоколом}]
    <<< MESSAGE
	destination:/topic/room/B73EAD/chatMessages
	content-type:application/json
	subscription:sub-2
	message-id:31jgkb5n-1
	content-length:306

	{"notificationType":"CHAT_MESSAGE","userId":"fb98d417-3ce4-46ed-9d50-e9c8ebdcdd15","text":"Hello World!","dateTime":{"dayOfWeek":"TUESDAY","dayOfYear":131,"nano":973629000,"year":2021,"monthValue":5,"dayOfMonth":11,"hour":13,"minute":13,"second":18,"month":"MAY","chronology":{"id":"ISO","calendarType":"iso8601"}}}
    \end{lstlisting}

    \registrationList

\end{document}